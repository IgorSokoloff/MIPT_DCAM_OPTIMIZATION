% Этот шаблон документа разработан в 2014 году
% Данилом Фёдоровых (danil@fedorovykh.ru) 
% для использования в курсе 
% <<Документы и презентации в \LaTeX>>, записанном НИУ ВШЭ
% для Coursera.org: http://coursera.org/course/latex .
% Исходная версия шаблона --- 
% https://www.writelatex.com/coursera/latex/2

\documentclass[a4paper,12pt]{article}

%%% Работа с русским языком
\usepackage{cmap}					% поиск в PDF
\usepackage{mathtext} 				% русские буквы в формулах
\usepackage[T2A]{fontenc}			% кодировка
\usepackage[utf8]{inputenc}			% кодировка исходного текста
\usepackage[english,russian]{babel}	% локализация и переносы
\usepackage{comment}


%%% Дополнительная работа с математикой
\usepackage{amsfonts,amssymb,amsthm,mathtools} % AMS
\usepackage{amsmath}
\usepackage{icomma} % "Умная" запятая: $0,2$ --- число, $0, 2$ --- перечисление

%% Номера формул
%\mathtoolsset{showonlyrefs=true} % Показывать номера только у тех формул, на которые есть \eqref{} в тексте.

%% Шрифты
\usepackage{euscript}	 % Шрифт Евклид
\usepackage{mathrsfs} % Красивый матшрифт

\usepackage{extsizes} % Возможность сделать 14-й шрифт
\usepackage{geometry} % Простой способ задавать поля
\geometry{top=25mm}
\geometry{bottom=35mm}
\geometry{left=20mm}
\geometry{right=20mm}

\usepackage{chngcntr}
\usepackage{hyperref}

\usepackage{setspace} % Интерлиньяж
%\onehalfspacing % Интерлиньяж 1.5
%\doublespacing % Интерлиньяж 2
%\singlespacing % Интерлиньяж 1

\usepackage{lastpage} % Узнать, сколько всего страниц в документе.
\usepackage{soulutf8} % Модификаторы начертания

\counterwithin*{equation}{section}
\counterwithin*{equation}{subsection}



%% Свои команды
\DeclareMathOperator{\sgn}{\mathop{sgn}}

%% Перенос знаков в формулах (по Львовскому)
\newcommand*{\hm}[1]{#1\nobreak\discretionary{}
{\hbox{$\mathsurround=0pt #1$}}{}}

%%% Работа с картинками
\usepackage{graphicx}  % Для вставки рисунков
\graphicspath{{images/}{images2/}}  % папки с картинками
\setlength\fboxsep{3pt} % Отступ рамки \fbox{} от рисунка
\setlength\fboxrule{1pt} % Толщина линий рамки \fbox{}
\usepackage{wrapfig} % Обтекание рисунков и таблиц текстом

%%% Работа с таблицами
\usepackage{array,tabularx,tabulary,booktabs} % Дополнительная работа с таблицами
\usepackage{longtable}  % Длинные таблицы
\usepackage{multirow} % Слияние строк в таблице
\usepackage{graphicx}
\usepackage{fancyhdr}
\usepackage{hyperref}
\usepackage{booktabs}

\newcommand{\lt}{\left}
\newcommand{\rt}{\right}

\pagestyle{fancy}
\fancyhf{}
\pagestyle{plain} % нумерация вкл.

\rhead{\today}
\lhead{Соколов Игорь, группа 573}

%%% Заголовок
\author{Соколов Игорь, группа 573}
\title{ДЗ 2 по Методам Оптимизации}
\date{\today}

\begin{document} % конец преамбулы, начало документа

\maketitle

\section{}

Покажите, что множество афинно тогда и только тогда, когда его пересечение с любой прямой афинно.

$X - aff \Leftrightarrow \forall a \rightarrow a\cap X - aff$

\begin{proof}
	
	\vspace{\baselineskip}
	
	$\Longrightarrow$
	
	Пусть $X - aff$.
	
	Любая прямая  - афинное множество по определению.
	
	Если $a\cap X = \varnothing$, тогда $a\cap X$ - $aff$ по опр.
	
	Если $a\cap X \neq \varnothing$, тогда возьмем $x_1, x_2 \in a\cap X $

	$\Rightarrow $

	$\begin{cases}
	x_1, x_2 \in a\\
	x_1, x_2 \in X\end{cases}
	$
	
	Так как $a - aff, X - aff$, то $\forall \theta \in \mathbb{R} \rightarrow $
	$\begin{cases} 
	x_{\theta} = \theta x_1+(1-\theta)x_2 \in a\\
	x_{\theta} = \theta x_1+(1-\theta)x_2 \in X
	\end{cases}
	$ 
	
	$\Rightarrow x_{\theta} \in a\cap X$
	
	$\Rightarrow a\cap X - aff$

	\vspace{\baselineskip}
	
	$\Longleftarrow$
		
	Пусть теперь $\forall a \rightarrow a\cap X - aff$. 
	
	Для произвольных $x_1,x_2 \in X \rightarrow$ пересечение $X$ и прямой $a$, содержащей $x_1,x_2$ афинно, то есть содержит прямую, проходящую через $x_1$ и $x_2 \Rightarrow a \in a\cap X$, так как через две точки можно провести толлко \pagestyle{plain} % нумерация вкл.одну прямую.
	
	Так как $a \in a\cap X \Rightarrow a \in X$
	
	То есть $\forall x_1,x_2 \in X \theta \in \mathbb{R}  \rightarrow \theta x_1+(1-\theta)x_2 \in X$
	
	$\Rightarrow X - aff$
\end{proof}

\section{}

Пусть $S_1, \ldots, S_k$ - произвольные непустые множества в $\mathbb{R}^n$. Докажите, что:

1. $ \mathbf{cone} \left( \bigcup\limits_{i=1}^k S_i\right) = \sum\limits_{i=1}^k \mathbf{cone} \left( S_i\right) $

2. $ \mathbf{conv} \left( \sum\limits_{i=1}^k S_i\right) = \sum\limits_{i=1}^k \mathbf{conv} \left( S_i\right) $

%2
Начнем с 2.

\begin{proof}

По индукции:

База: $k = 1 \Rightarrow \mathbf{conv} \left( \sum\limits_{i=1}^1 S_i\right) = \sum\limits_{i=1}^1 \mathbf{conv}\left( S_i\right) = \mathbf{conv} \left( S_1\right)$- верно.

Пусть верно для $k = n \Rightarrow \mathbf{conv} \left( \sum\limits_{i=1}^n S_i\right) = \sum\limits_{i=1}^n \mathbf{conv} \left( S_i\right) $ - верно.

Докажем для $k = n+1$, то есть докажем равенство: 

$$\mathbf{conv} \left( \sum\limits_{i=1}^{n+1} S_i\right) = \sum\limits_{i=1}^{n+1} \mathbf{conv} \left( S_i\right)$$

$$\mathbf{conv} \left( \sum\limits_{i=1}^{n} S_i + S_{n+1}\right) = \mathbf{conv} \left(\sum\limits_{i=1}^{n} S_i\right) + \mathbf{conv}\left( S_{n+1}\right)$$

Переобозначим:

$A = \sum\limits_{i=1}^{n} S_i$

$B = S_{n+1}$

Тогда $$\mathbf{conv}\left( A +B\right) = \mathbf{conv}\left( A \right)+ \mathbf{conv}\left( B\right)$$

$\Longrightarrow $

$A \subset \mathbf{conv}\left( A \right)$

$B \subset \mathbf{conv}\left( B \right)$

$A + B \subset \mathbf{conv}\left( A\right) + \mathbf{conv}\left( B\right)$

Так как $\mathbf{conv}\left( A\right) + \mathbf{conv}\left( B\right) - convex$ (так как сумма выпуклых множеств).

То $\mathbf{conv}\left( A + B\right)\subset\mathbf{conv}\left( A\right) + \mathbf{conv}\left( B\right)$ ( так как $\mathbf{conv}\left( A + B\right)$ минимальное выпуклое множество, такое что $A+B \subseteq \mathbf{conv}\left( A + B\right)$)

\vspace{\baselineskip}
\textbf{Что значит "минимальное"?}

Так как $\mathbf{conv}\left( A + B\right)$ есть пересечение всевозможных выпуклых множеств, содержащих $A+B$, то есть $\bigcap\limits_{i=0}^n S_i = \mathbf{conv}\left( A + B\right)$.

Можно ввести порядок по вложению: $S_{i_1}\subseteq S_{i_2}\subseteq \dots\subseteq S_{i_n}\subseteq A+B, \text{где } i_j \in [1,n]$ 

И под \textbf{"минимальным"} множеством подразумевалось как раз $S_{i_1}$


~\

$\Longleftarrow $

Пусть $\quad z \in \mathbf{conv}\left( A\right) + \mathbf{conv}\left( B\right),\quad$ то есть $z = x +y,\quad$ где $x \in \mathbf{conv}\left(A\right), y \in \mathbf{conv}\left(B\right)$ 

Тогда $x = \sum\limits_{i=1}^{l} \alpha_ix_i , y = \sum\limits_{j=1}^{p} \beta_jy_j,\quad \alpha_i \geq 0, \beta_j \geq 0,\quad \sum\limits_{i=1}^{l} \alpha_i = \sum\limits_{j=1}^{p} \beta_j = 1$

$x + y \in \mathbf{conv}\left( A\right) + \mathbf{conv}\left( B\right) $

Для начала, заметим, что $x + y_j \in \mathbf{conv}(A+B)$, так как  $x + y_j = \sum\limits_{i=1}^{l} \alpha_i(x_i+y_j)$

Затем можно записать $x+y$ как выпуклую комбинацию точек $x+y_j$, а именно:

$$z = x + y =\sum\limits_{j=1}^{p} \beta_j(x+y_j) = \sum\limits_{j=1}^{p}\sum\limits_{i=1}^{l} \alpha_i\beta_j(x_i+y_j)$$

Где $x_i + y_i \in A+ B, \alpha_i\beta_j\geq 0,\sum\limits_{j=1}^{p}\sum\limits_{i=1}^{l} \alpha_i\beta_j =\sum\limits_{i=1}^{l}\alpha_i \sum\limits_{j=1}^{p}\beta_j =1$

Таким образом $z$ является выпуклой комбинацией точек множества $A+B$

$\Rightarrow z \in \mathbf{conv}\left( A + B\right)$

Так как $z$ было выбрано произвольно, то верно $  \mathbf{conv}\left( A\right) + \mathbf{conv}\left( B\right) \subset \mathbf{conv}\left( A + B\right)$

Из включения в обе стороны следует равенство:
$$  \mathbf{conv}\left( A + B\right) = \mathbf{conv}\left( A\right) + \mathbf{conv}\left( B\right) $$

\end{proof}
Доказательство 1 идейно повторяет док-во 2, за исключением некоторых деталей.

\begin{proof}
По индукции:

База: $k = 1 \Rightarrow \mathbf{cone} \left( \bigcup\limits_{i=1}^1 S_i\right) = \sum\limits_{i=1}^1 \mathbf{cone}\left( S_i\right) = \mathbf{cone} \left( S_1\right)$- верно.

Пусть верно для $k = n \Rightarrow \mathbf{cone} \left( \bigcup\limits_{i=1}^n S_i\right) = \sum\limits_{i=1}^n \mathbf{cone} \left( S_i\right) $ - верно.

Докажем для $k = n+1$, то есть докажем равенство: 

$$\mathbf{cone} \left( \bigcup\limits_{i=1}^{n+1} S_i\right) = \sum\limits_{i=1}^{n+1} \mathbf{cone} \left( S_i\right)$$

$$\mathbf{cone} \left( \bigcup\limits_{i=1}^{n} S_i \cup S_{n+1}\right) = \mathbf{cone} \left(\bigcup\limits_{i=1}^{n} S_i\right) + \mathbf{cone}\left( S_{n+1}\right)$$

Переобозначим:

$A = \bigcup\limits_{i=1}^{n} S_i$

$B = S_{n+1}$

Тогда $$\mathbf{cone}\left( A \cup B\right) = \mathbf{cone}\left( A \right)+ \mathbf{cone}\left( B\right)$$

$\Longrightarrow $

$A \subset \mathbf{cone}\left( A\right)$

$B \subset \mathbf{cone}\left( B\right)$

По опр:

$\mathbf{cone}(A) = \left\{ \sum\limits_{i=1}^l\theta_i x_i \mid x_i \in A, \theta_i \ge 0\right\}$

$\mathbf{cone}(B) = \left\{ \sum\limits_{j=1}^p\lambda_i y_i \mid y_j \in B, \lambda_j \ge 0\right\}$

Тогда 
$\mathbf{cone}(A \cup B) = \left\{ \sum\limits_{i=1}^{l+p}\theta_i z_i   \mid z_i \in A\cup B, \theta_i \ge 0\right\}$

Так как, $A\cup B$ состоит из $x$, $y$(как определено выше)

$\sum\limits_{i=1}^{l+p}\theta_i z_i = \sum\limits_{i=1}^l\theta_i x_i + \sum\limits_{i=l+1}^{l+p}\theta_i y_i$

\vspace{\baselineskip}
\vspace{\baselineskip}
\vspace{\baselineskip}
Переобозначим 

$\theta_{l+1} \rightarrow \lambda_1$

$\theta_{l+2} \rightarrow \lambda_2$

$\dots$

$\theta_{l+p} \rightarrow \lambda_p$
\vspace{\baselineskip}

$y_{l+1} \rightarrow y_1$

$y_{l+2} \rightarrow y_2$

$\dots$

$y_{l+p} \rightarrow y_p$

Тогда можно записать:

\begin{equation}\label{formula1}
\mathbf{cone}(A \cup B) = \left\{ \sum\limits_{i=1}^l\theta_i x_i + \sum\limits_{j=1}^{p}\lambda_j y_j  \mid x_i \in A, y_j \in B, \theta_i, \lambda_j \ge 0\right\}
\end{equation}


Так как в формулу \hyperref[formula1]{(1)} входят всевозможные $\lambda_j, \theta_i \in \mathbb{R}$, то её можно представить как сумму Минковского всевозможных конических комбинаций, то есть:

\begin{multline}
\left\{ \sum\limits_{i=1}^l\theta_i x_i + \sum\limits_{j=1}^{p}\lambda_j y_j  \mid x_i \in A, y_j \in B, \theta_i, \lambda_j \ge 0\right\} = \left\{ \sum\limits_{i=1}^l\theta_i x_i \mid x_i \in A, \theta_i \ge 0\right\} +\\+
\left\{\sum\limits_{j=1}^p\lambda_i y_i \mid y_j \in B, \lambda_j \ge 0\right\} 
\end{multline}

Что совпадает с суммой конических оболочек $A$ и $B$.
\begin{equation}\label{formula2}
\mathbf{cone}(A) + \mathbf{cone}(B) = \left\{ \sum\limits_{i=1}^l\theta_i x_i \mid x_i \in A, \theta_i \ge 0\right\} +
\left\{\sum\limits_{j=1}^p\lambda_i y_i \mid y_j \in B, \lambda_j \ge 0\right\}
\end{equation}

$\Rightarrow \mathbf{cone}(A \cup B) = \mathbf{cone}(A) + \mathbf{cone}(B)$

\end{proof}

%3

\section{}

Докажите, что множество $S \subseteq \mathbb{R}^n$ выпукло тогда и только тогда, когда $(\alpha + \beta)S = \alpha S + \beta S$ для всех неотрицательных $\alpha$ и $\beta$

$S - convex \Leftrightarrow \forall \alpha,\beta \geq 0 \rightarrow (\alpha + \beta)S = \alpha S + \beta S$

\begin{proof}

$\Longrightarrow$

Так как $S - convex$, то $\forall \alpha,\beta \geq 0 \rightarrow \alpha S + \beta S - convex$
\vspace{\baselineskip}

1. Если $\alpha = \beta = 0$

$ (\alpha + \beta)S = (0+0)S = 0$
 
$\alpha S + \beta S = 0S + 0S = 0$

$\Rightarrow  (\alpha + \beta)S = \alpha S + \beta S = 0$
\vspace{\baselineskip}

2. Если $\alpha=0, \beta \neq 0$(для случая $\beta = 0$ аналогично)

$(\alpha + \beta)S = (0 + \beta)S = \beta S$

$\alpha S + \beta S = 0S + \beta S = \beta S$

$\Rightarrow  (\alpha + \beta)S = \alpha S + \beta S = \beta S$

\vspace{\baselineskip}
3. Если $\alpha \neq 0, \beta \neq0$ поделим на $(\alpha + \beta)$

\[
\lt(\frac{\alpha}{\alpha+\beta}\rt)S + \lt(1 - \frac{\alpha}{\alpha+\beta}\rt)S=S
\]

Пусть $ \frac{\alpha}{\alpha+\beta} = \theta \in [0;1]$

Тогда надо доказать, что 
\[
\theta S + \lt(1 - \theta\rt)S=S
\]


\begin{equation}
\theta S + \lt(1 - \theta\rt)S = \lt\{x_\theta \mid x_\theta =  \theta x_1+\lt(1-\theta\rt)x_2,\quad \forall x_1,x_2 \in S, \theta \in [0;1] \rt\}
\end{equation}

Так как $S$ - выпукло, то $\forall x_\theta \rightarrow x_\theta \in S$ 

$\Rightarrow \theta S + \lt(1 - \theta\rt)S=S$ - верно.

$\Rightarrow (\alpha + \beta)S = \alpha S + \beta S$ - верно.

\vspace{\baselineskip}

$\Longleftarrow$

По условию имеем, что  $(\alpha + \beta)S = \alpha S + \beta S$ - верно.

При $\alpha,\beta \neq 0 $ получаем $\theta S + \lt(1 - \theta\rt)S=S$ (случаи когда $\alpha =0$ и (или) $\beta =0$ рассматриваются аналогично) 
 
\[
\theta S + \lt(1 - \theta\rt)S = S \Leftrightarrow \lt\{\theta x_1+\lt(1-\theta\rt)x_2 = x_\theta \mid \forall x_1,x_2 \in S,x_\theta \in S, \theta \in [0;1] \rt\}
\]

Откуда следует выполнение определения выпуклого множества.

$\Rightarrow S$ - выпукло.
 

\begin{comment}
Пусть $A = \alpha S$

$B = \beta S$

Тогда надо доказать, что $X + Y = Z - convex$

$Z = \left\{z \mid z = \alpha x + \beta y, x \in X, y \in Y, \alpha, \beta \in \mathbb{R}^1_+\right\}$
 
Возьмем две точки из $Z$: $z_1 = \alpha_1 x_1 + c_2 y_1, z_2 = c_1 x_2 + c_2 y_2$ и докажем, что отрезок между ними $\theta s_1 + (1 - \theta)s_2, \theta \in [0,1]$ так же принадлежит $S$
 
 $$\theta s_1 + (1 - \theta)s_2$$
 
 $$\theta (c_1 x_1 + c_2 y_1) + (1 - \theta)(c_1 x_2 + c_2 y_2)$$
 
 $$c_1 (\theta x_1 + (1 - \theta)x_2) + c_2 (\theta y_1 + (1 - \theta)y_2)$$
 
 $$c_1 x + c_2 y \in S$$
\end{comment}

	
\end{proof}

%4
\section{}
Пусть $x \in \mathbb{R}$ - случайная величина с заданным вероятностным распределением $\mathbb{P}(x = a_i) = p_i$, где $i = 1, \ldots, n$, а $a_1 < \ldots < a_n$. Говорят, что вектор вероятностей исходов $p \in \mathbb{R}^n$ принадлежит вероятностному симплексу, т.е. $P = \left\{ p \mid \mathbf{1}^Tp = 1, p \succeq 0\right\} = \left\{ p \mid p_1 + \ldots + p_n = 1, p_i \ge 0 \right\}$. 

Определите, выпукло ли множество таких $p$, которые удовлетворяют условию:

1. $\mathbb{P}(x > \alpha) \le \beta$

2. $\mathbb{E} |x^{2017}| \le \alpha \mathbb{E}|x|$

3. $\mathbb{E} |x^{2}| \ge \alpha $

4. $\mathbb{V}x \ge \alpha$

\begin{proof}

1) Заметим, что данное неравенство эквивалентно линейному неравенству:

$$\sum_{i: a_i>\alpha}p_i\leq\beta$$

Которое задает полупространство $A$ - $convex$.

Вероятностный симплекс $P$ - $convex$

$\Rightarrow A \cap P - convex$

2)
$$\sum_{i=1}^{n}p_i\left(\left|a_i\right|^{2017}-\alpha\left|a_i\right|\right)\leq 0$$

Пусть $\left(|a_i|^{2017}-\alpha|a_i|\right) = \beta_i$

$$\sum_{i=1}^{n}p_i\beta_i\leq 0$$
Также задает некоторое полупространство($convex$), которое при пересечении с симплексом дает выпуклое множество.


3) $$\sum_{i=1}^{n}p_ia_i^2\geq\alpha$$

Также линейное неравенство, задающее некоторое полупространство и, пересекая его с симплексом, снова получаем выпуклое множество

4) $\mathbb{V}x = \mathbb{E}\left\{ (x - \mathbb{E}x)^2\right\} = \mathbb{E}x^2 - \left(\mathbb{E}x\right)^2$ = $\sum\limits_{i=1}^n p_i a_i^2 - \left(\sum\limits_{i=1}^n p_i a_i\right)^2 \ge \alpha$
	
Перейдем к другим координатам.

$p_i a_i = y_i$ (при аффинном отображении сохраняется выпуклость).
 	
$\sum\limits_{i=1}^n y_i a_i - \left(\sum\limits_{i=1}^n y_i\right)^2 \ge \alpha$

$\left(\sum\limits_{i=1}^n y_i\right)^2 - \sum\limits_{i=1}^n y_i a_i \le \alpha$

Слева от неравенства некоторый квадратичный функционал, который можно привести к каноническому виду и далее увидеть, что полученная каноничная форма есть выпуклая фигура.

Перечение с симплексом даст снова выпуклое множество.
\end{proof}

\end{document} % конец документа

