\documentclass[a4paper,12pt]{article}

%%% Работа с русским языком
\usepackage{cmap}					% поиск в PDF
\usepackage{mathtext} 				% русские буквы в формулах
\usepackage[T2A]{fontenc}			% кодировка
\usepackage[utf8]{inputenc}			% кодировка исходного текста
\usepackage[english,russian]{babel}	% локализация и переносы
\usepackage{comment}


%%% Дополнительная работа с математикой
\usepackage{amsfonts,amssymb,amsthm,mathtools} % AMS
\usepackage{amsmath}
\usepackage{icomma} % "Умная" запятая: $0,2$ --- число, $0, 2$ --- перечисление

%% Номера формул
%\mathtoolsset{showonlyrefs=true} % Показывать номера только у тех формул, на которые есть \eqref{} в тексте.

%% Шрифты
\usepackage{euscript}	 % Шрифт Евклид
\usepackage{mathrsfs} % Красивый матшрифт

\usepackage{extsizes} % Возможность сделать 14-й шрифт
\usepackage{geometry} % Простой способ задавать поля
\geometry{top=25mm}
\geometry{bottom=35mm}
\geometry{left=20mm}
\geometry{right=20mm}

\usepackage{chngcntr}
\usepackage{hyperref}

\usepackage{setspace} % Интерлиньяж
%\onehalfspacing % Интерлиньяж 1.5
%\doublespacing % Интерлиньяж 2
%\singlespacing % Интерлиньяж 1

\usepackage{lastpage} % Узнать, сколько всего страниц в документе.
\usepackage{soulutf8} % Модификаторы начертания

\counterwithin*{equation}{section}
\counterwithin*{equation}{subsection}



%% Свои команды
\DeclareMathOperator{\sgn}{\mathop{sgn}}

%% Перенос знаков в формулах (по Львовскому)
\newcommand*{\hm}[1]{#1\nobreak\discretionary{}
{\hbox{$\mathsurround=0pt #1$}}{}}

%%% Работа с картинками
\usepackage{graphicx}  % Для вставки рисунков
\graphicspath{{images/}{images2/}}  % папки с картинками
\setlength\fboxsep{3pt} % Отступ рамки \fbox{} от рисунка
\setlength\fboxrule{1pt} % Толщина линий рамки \fbox{}
\usepackage{wrapfig} % Обтекание рисунков и таблиц текстом

%%% Работа с таблицами
\usepackage{array,tabularx,tabulary,booktabs} % Дополнительная работа с таблицами
\usepackage{longtable}  % Длинные таблицы
\usepackage{multirow} % Слияние строк в таблице
\usepackage{graphicx}
\usepackage{fancyhdr}
\usepackage{hyperref}
\usepackage{booktabs}

\newcommand{\lt}{\left}
\newcommand{\rt}{\right}
\newcommand{\al}{\alpha}
\newcommand{\p}{\partial}
\newcommand{\D}{\Delta}
\newcommand{\fr}{\frac}
\newcommand{\dfr}{\dfrac}
\newcommand{\mbf}{\mathbf}
\newcommand{\bb}{\mathbb}
\newcommand{\wt}{\widetilde}
\newcommand{\La}{\Lambda}
\newcommand{\la}{\lambda}
\newcommand{\opn}{\operatorname}


\pagestyle{fancy}
\fancyhf{}
\pagestyle{plain} % нумерация вкл.

\rhead{\today}
\lhead{Соколов Игорь, группа 573}

%%% Заголовок
\author{Соколов Игорь, группа 573}
\title{ДЗ 6 по Методам Оптимизации. \newline Выпуклые функции. Сильно выпуклые функции.}
\date{\today}

\begin{document} % конец преамбулы, начало документа

\maketitle



\section{}

Выпуклы ли следующие функции: 

1. $f(x) = e^x - 1, \; x \in \mathbb{R};\;\;\; $

2. $f(x_1, x_2) = x_1x_2, \; x \in \mathbb{R}^2_{++};\;\;\; $

3. $f(x_1, x_2) = 1/(x_1x_2), \; x \in \mathbb{R}^2_{++}$?

\vspace{\baselineskip}

\textbf{Решение:}

\vspace{\baselineskip}

\subsection{}
 $f(x) = e^x - 1$, тогда $f''(x) = e^x > 0$ на $R \Rightarrow$ по дифференциальному критерию 2-го порядка $f$ выпукла на $R$.


\subsection{}
$f(x_1, x_2) = x_1x_2$. 

$$
f''(x) = 
\begin{pmatrix}
0 & 1 \\
1 & 0 \\
\end{pmatrix}.
$$

Эта форма не является положительно определенной. 

$\Rightarrow f(x_1,x_2)$  не выпукла.

\subsection{}

$f(x_1, x_2) = 1/(x_1 x_2)$.
$$
f''(x) = 
\begin{pmatrix}
\dfr{2}{x_1^3x_2} & \dfr{1}{x_1^2x_2^2} \\
\dfr{1}{x_1^2x_2^2} & \dfr{2}{x_1x_2^3} \\
\end{pmatrix}.
$$

$\Delta_1 = \dfr{2}{x_1^3x_2} > 0$ и $\Delta_2= \det(f''(x)) = \dfr{3}{x_1^4x_2^4} > 0$ на $R^2_{++}$. 

Значит $f''(x)$ положительно определена $\Rightarrow$ по критерию 2-го порядка $f$ выпукла на $R^2_{++}$


\section{}

Докажите, что множество $S = \left\{ x \in \mathbb{R}^n_{++} \mid \prod\limits_{i=1}^n x_i \geq 1 \right\}$ выпукло.

\vspace{\baselineskip}

\begin{proof}

$$S = \left\{ x \in \mathbb{R}^n_{++} \mid \prod\limits_{i=1}^n x_i \geq 1 \right\} =\left\{ x \in \mathbb{R}^n_{++} \mid x_n \geq \fr{1}{\prod\limits_{i=1}^{n-1}x_i}
 \right\} $$

$\Rightarrow S - $ надграфик функции $f(x_1,\dots, x_{n-1}) = \dfr{1}{\prod\limits_{i=1}^{n-1}x_i}$

$$\fr{\p f}{\p x_k} = -\fr{1}{x_k^2 \prod\limits_{i = 1, i \neq k}^{n-1}x_i}$$

$$\fr{\p^2 f}{\p x_k x_k} = \fr{2}{x_k^3 \prod\limits_{i = 1, i \neq k}^{n-1}x_i} = \fr{2}{x_k^2 \prod\limits_{i = 1}^{n-1}x_i}  = \fr{2f}{x_k^2}$$

$$\fr{\p^2 f}{\p x_k x_p} = \fr{1}{x_k^2 x_p^2 \prod\limits_{i = 1, i \neq k, p }^{n-1}x_i}= \fr{2}{x_k x_p \prod\limits_{i = 1}^{n-1}x_i}  = \fr{2f}{x_k x_p}$$


$$f''(x) = 2f
\begin{pmatrix}
\dfr{1}{x_1^2} & \dfr{1}{x_1 x_2}& \dfr{1}{x_1 x_3}&\dfr{1}{x_1 x_4}&\dots& \dfr{1}{x_1 x_n}\\
\dfr{1}{x_2 x_1} & \dfr{1}{x_2^2}& \dfr{1}{x_2 x_3}&\dfr{1}{x_2 x_4}&\dots& \dfr{1}{x_2 x_n}\\
\dfr{1}{x_3 x_1} & \dfr{1}{x_3 x_2}& \dfr{1}{x_3^2}&\dfr{1}{x_3 x_4}&\dots& \dfr{1}{x_3 x_n}\\
\dfr{1}{x_4 x_1} & \dfr{1}{x_4 x_2}& \dfr{1}{x_4 x_3}&\dfr{1}{x_4^2}&\dots& \dfr{1}{x_4 x_n}\\
\vdots&\vdots&\vdots&\vdots&\ddots&\vdots\\
\dfr{1}{x_n x_1} & \dfr{1}{x_n x_2}& \dfr{1}{x_n x_3}&\dfr{1}{x_n x_4}&\dots& \dfr{1}{x_n ^2}\\
\end{pmatrix}.$$

$\Delta_1 = 
\begin{vmatrix}
\dfr{1}{x_1^2} & \dfr{1}{x_1 x_2}\\
\dfr{1}{x_2 x_1} & \dfr{1}{x_2^2}\\
\end{vmatrix} = 0$

\begin{multline}
\Delta_2 =  
\begin{vmatrix}
\dfr{1}{x_1^2} & \dfr{1}{x_1 x_2}& \dfr{1}{x_1 x_3}\\
\dfr{1}{x_2 x_1} & \dfr{1}{x_2^2}& \dfr{1}{x_2 x_3}\\
\dfr{1}{x_3 x_1} & \dfr{1}{x_3 x_2}& \dfr{1}{x_3^2}\\
\end{vmatrix} 
=
\dfr{1}{x_1^2}
\begin{vmatrix}
\dfr{1}{x_2^2}& \dfr{1}{x_2 x_3}\\
\dfr{1}{x_3 x_2}& \dfr{1}{x_3^2}\\
\end{vmatrix}
-
\dfr{1}{x_1 x_2}
\begin{vmatrix}
\dfr{1}{x_2 x_1} &\dfr{1}{x_2 x_3}\\
\dfr{1}{x_3 x_1} &	\dfr{1}{x_3^2}\\
\end{vmatrix}
+
\dfr{1}{x_1 x_3}
\begin{vmatrix}
\dfr{1}{x_2 x_1} & \dfr{1}{x_2^2}\\
\dfr{1}{x_3 x_1} & \dfr{1}{x_3 x_2}\\
\end{vmatrix}
=\\= \dfr{1}{x_1^2} \cdot 0 - \dfr{1}{x_1 x_2} \cdot 0 + \dfr{1}{x_1 x_3}\cdot 0 = 0
\end{multline}


И так далее можно показать, что $\forall i \rightarrow \Delta_i = 0$ 

\vspace{\baselineskip}

{\bf То есть:}
Определитель матрицы $n\times n$ можно подсчитать разложением по строке. 
После одного разложения надо подсчитать сумму $n$ определителей матриц размера $n-1\times n-1$


Применим эту операцию рекурсивно для подсчета определителей каждой из матриц(рекурсивно опускаемся вниз матрицы).

После проделывания этой процедуры $n-2$ раза ($n-2$ разложения по строке), необходимо будет подсчитать определители вида  
$\begin{vmatrix} 
\dfr{1}{x_{n-1} x_j} & \dfr{1}{x_{n-1} x_{j+k}}\\
\dfr{1}{x_{n} x_j} & \dfr{1}{x_{n} x_{j+k}}\\
\end{vmatrix} = 0 \quad \forall j \in [1, n]$

Затем поднимаясь вверх по рекурсии получим увидем, что $\forall i \rightarrow \Delta_i = 0$

\vspace{\baselineskip}

$\Rightarrow f''(x) $ положительно полуопределенная матрица  

$\Rightarrow $ по критерию 2-го порядка $f(x)$ выпуклая функция.

$\Rightarrow $ Надграфик $f(x)$ - выпуклое множество.

$\Rightarrow S - $ выпуклое множество.

\end{proof}
%3

\section{}

Докажите, что функция 

1. $f(X) = \mathbf{tr}(X^{-1}), X \in S^n_{++}$ выпукла;

2. $f(X) = (\det X)^{1/n}, X \in S^n_{++}$ вогнута.

\subsection{}
\begin{proof}
Пусть $g(t) =  {\bf tr}(Z + tV)$, где  $Z\succ 0$ и $V \in {\bf S}^n_{++} $

\begin{multline}\label{tr_eq}
g(t) = {\bf tr}(\lt(Z + tV)^{-1}\rt) 
=\\= 
{\bf tr}\lt((Z^{-1}(\bb E + tZ^{-1/2} V Z^{-1/2})^{-1}\rt) 
=\\= 
{\bf tr}\lt((Z^{-1}Q(\bb E + t\La)^{-1}Q^T\rt)
=\\= 
{\bf tr}\lt(Q^TZ^{-1}Q(\bb E + t\La)^{-1}\rt)
=\\=
\sum \limits_{i=1}^{n} \fr{\lt(Q^TZ^{-1}Q \rt)_{ii}}{1 + t\lambda_i}
\end{multline}

В серии равенств выше мы воспользовались следующим:

\begin{itemize} 
	\item $Z^{-1/2}VZ^{-1/2} = Q\La Q^{-1} $, где $Q$ - матрица перехода к базису из собственных векторов.
	\item ${\bf tr}( \bb E + t\La)^{-1} = \sum \limits_{i=1}^{n}\dfr{1}{1 + t\lambda_i}$
	\item ${\bf tr}(Q^TZ^{-1}Q( \bb E + t\La)^{-1}) = {\bf tr}(Q^TZ^{-1}Q\lt[( \bb E + t\La)^{-1}\rt]^T) = \sum\limits_{i}\dfr{\lt(Q^TZ^{-1}Q \rt)_{ii}}{1 + t\lambda_i}$, так как ${\bf tr}(XY^T) = \sum\limits_{i,j}X_{ij}Y_{ij}$  и  $\lt(\bb E + t\La\rt)^{-1}$ диагональна
\end{itemize}	

В последнем равенстве выражения \eqref{tr_eq} видим неотрицательную комбинацию выпуклых функций $\phi(t) = \dfr{1}{1 + \la_i t}$ (их выпуклость легко показать, через критерий 2-го порядка)

$\Rightarrow f(X) - $ выпуклая функция.
\end{proof}	
	
\subsection{}

\begin{proof}
Пусть $g(t) =  f(Z + tV)$, где  $Z\succ 0$ и $V \in {\bf S}^n_{++} $

$$
(\det (Z + Vt))^{1/n} = (\det Z)^{1/n} (\det (I + tZ^{-1/2}VZ^{-1/2}))^{1/n} = (\det Z)^{1/n} \left(\prod\limits_{i=1}^n (1 + t \lambda_i)\right)^{1/n}.
$$

Где $\la_i$ - собственные значения матрицы $Z^{-1/2}VZ^{-1/2}$ и $det(Z) > 0$

Легко показать по критерию 2-го порядка, что $\left(\prod\limits_{i=1}^n (1 + t \lambda_i)\right)^{1/n}$ вогнутая функция.

Действительно:

$ \dfr{\p^2 f}{\p x_i \p x_j} =\dfrac{f}{n^2}A \quad \text{где  } \ A_{ij}= \begin{cases}(1-n)x_i^{-2} \quad &\text{ if }\ i=j \\
x_i^{-1}x_j^{-1} \quad &\text{ if }\ i\ne j \end{cases} $

Пусть $y_i = 1/x_i$. Нам надо доказать, что $v^TAv \le 0 \forall v$

Действительно:
$v^TAv=\left(\sum_{i=1}^n y_iv_i\right)^2-n \sum_{i=1}^n y_i^2 v_i^2 \le 0$ по формуле Коши-Буняковского примененного к $1\times (y_iv_i)$

$\Rightarrow $ показали вогнутость функции $\left(\prod\limits_{i=1}^n (1 + t \lambda_i)\right)^{1/n}$

$\Rightarrow f(X) - $ вогнутая.

\end{proof}
%4
\section{}

Расстоянием Кульбака - Лейблера между между $p,q \in \mathbb{R}^n_{++}$ называется:
 $$
 D(p,q) = \sum\limits_{i=1}^n (p_i \log(p_i/q_i) - p_i + q_i)
 $$
 Докажите, что $D(p,q) \geq 0 \forall p,q \in \mathbb{R}^n_{++}$ и $D(p,q) = 0 \leftrightarrow p = q$  
 Подсказка:  $$
 D(p,q) = f(p) - f(q) - \nabla f(q)^T(p-q), \;\;\;\; f(p) = \sum\limits_{i=1}^n p_i \log p_i
 $$


\begin{proof}

\begin{equation}\label{eq_plogp}
f(p) = \sum\limits_{i=1}^n p_i \log p_i
\end{equation}
$$\fr{\p f}{\p x_k} = 1 + \log p_k$$
$$\fr{\p^2 f}{\p x_k^2} = \dfr{1}{p_k} > 0 \quad \forall p \in \bb R^n_{++}$$

$\Rightarrow f(p)$ - строго выпуклая функция.

\begin{equation}\label{eq_pq}
\Rightarrow f(p) > f(q) + \nabla f(q)^T(p - q ) \quad \forall p,q \in \bb R^n_{++}: p \neq q \quad \text{ критерий 1-го порядка}
\end{equation}

$$\sum\limits_{i=1}^n p_i \log p_i > \sum\limits_{i=1}^n q_i \log q_i + \sum\limits_{i=1}^n (1 + \log q_i)(p_i - q_i) =  \sum\limits_{i=1}^n \lt(p_i\log q_i + p_i - q_i\rt)$$
$$\sum\limits_{i=1}^n p_i \log p_i - \sum\limits_{i=1}^n \lt(p_i\log q_i - p_i + q_i\rt) > 0$$
$$\sum\limits_{i=1}^n\lt( p_i \log (p_i/q_i) - p_i + q_i\rt) > 0$$
$$\Rightarrow D(p, q) > 0 , p\neq q$$

Так как $D(p, q) = f(p) - f(q) - \nabla f(q)^T(p - q) \Leftrightarrow D(p, q) = 0 \text{ при } p = q $  

	
	
\end{proof}

%5
\section{}

Пусть $x$ - действительнозначная переменная, принимающая конечный набор значений $a_1 < a_2 < \ldots < a_n$ с вероятностями $\mathbb{P}(x = a_i) = p_i$. Оцените выпуклость и вогнутость следующих функций от $p$ на множестве $\left\{p \mid \sum\limits_{i=1}^n p_i = 1, p_i \ge 0 \right\}$  

1. $\mathbb{E}x$

2. $\mathbb{P}\{x \ge \alpha\}$

3. $\mathbb{P}\{\alpha \le x \le \beta\}$

4. $\sum\limits_{i=1}^n p_i \log p_i$

5. $\mathbb{V}x = \mathbb{E}(x - \mathbb{E}x)^2$

6. $\mathbf{quartile}(x) = {\operatorname{inf}}\left\{ \beta \mid \mathbb{P}\{x \le \beta\} \ge 0.25 \right\}$

\vspace{\baselineskip}

\textbf{Решение:}

\begin{enumerate}
\item 
$\bb Ex = p_1a_1 + \dots + p_na_n - $ неотрицательная комбинация константных функций $a_i$

$\Rightarrow \bb Ex -$ выпукла и вогнута.

\item 	
Пусть $k  = min\{i \mid a_i \ge \al\}$. Тогда $\bb P(x \ge \al) = \sum \limits_{i = k}^n p_i.$ - линейная функция от переменной $\bf p \Rightarrow$ выпукла и вогнута (по критерию 2-го порядка, так как $f''(p) = 0$).

\item 
Пусть $k  = min\{i \mid a_i \ge \al\}$ и $m = max\{i\mid a_i \le \beta\}$. 

Тогда $\mathbb{P}\{\alpha \le x \le \beta\} = \sum \limits_{i = k}^m p_i$ - линейная функция от переменной $\bf p \Rightarrow$ выпукла и вогнута.

\item В предыдущей задаче \eqref{eq_plogp} было показано, что $\sum\limits_{i=1}^n p_i \log p_i$ - строго выпуклая.

$\Rightarrow$ не является вогнутой.
\item $\mathbb{V}x = \mathbb{E}(x - \mathbb{E}x)^2 = \bb E$

\item $$\mathbf{quartile}(x) = {\opn{inf}}\left\{ \beta \mid \mathbb{P}\{x \le \beta\} \ge 0.25 \right\}$$ $$= {\opn{inf}}\lt\{ \beta \mid 1 - \mathbb{P}\{\beta < x \} \ge 0.25 \rt\}$$ $$= {\opn{inf}}\lt\{ \beta \mid \mathbb{P}\{\beta < x \} \le 0.75 \rt\}$$


Заметим, что функция $f(x) = {\bf quartile}(x)$ не является непрерывной (принимает значения из дискретного множества $\{a_1,\dots, a_n\}$). 

По теореме 3.1.5 из Жадана (стр 100): если $f(x)$ выпуклая функция на выпуклом множестве $X\subseteq \bb R^n$, то  $f(x)$ непрерывна $\forall x \in ri(X)$.

Значит $f(x) = {\bf quartile}(x)$ не является выпуклой. Заметим,что $-f(x)$ тоже не является выпуклой, значит $f(x)$ не является вогнутой.

\end{enumerate}

\end{document} % конец документа

