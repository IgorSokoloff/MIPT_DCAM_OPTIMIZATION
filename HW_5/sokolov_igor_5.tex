\documentclass[a4paper,12pt]{article}

%%% Работа с русским языком
\usepackage{cmap}					% поиск в PDF
\usepackage{mathtext} 				% русские буквы в формулах
\usepackage[T2A]{fontenc}			% кодировка
\usepackage[utf8]{inputenc}			% кодировка исходного текста
\usepackage[english,russian]{babel}	% локализация и переносы
\usepackage{comment}


%%% Дополнительная работа с математикой
\usepackage{amsfonts,amssymb,amsthm,mathtools} % AMS
\usepackage{amsmath}
\usepackage{icomma} % "Умная" запятая: $0,2$ --- число, $0, 2$ --- перечисление

%% Номера формул
%\mathtoolsset{showonlyrefs=true} % Показывать номера только у тех формул, на которые есть \eqref{} в тексте.

%% Шрифты
\usepackage{euscript}	 % Шрифт Евклид
\usepackage{mathrsfs} % Красивый матшрифт

\usepackage{extsizes} % Возможность сделать 14-й шрифт
\usepackage{geometry} % Простой способ задавать поля
\geometry{top=25mm}
\geometry{bottom=35mm}
\geometry{left=20mm}
\geometry{right=20mm}

\usepackage{chngcntr}
\usepackage{hyperref}

\usepackage{setspace} % Интерлиньяж
%\onehalfspacing % Интерлиньяж 1.5
%\doublespacing % Интерлиньяж 2
%\singlespacing % Интерлиньяж 1

\usepackage{lastpage} % Узнать, сколько всего страниц в документе.
\usepackage{soulutf8} % Модификаторы начертания

\counterwithin*{equation}{section}
\counterwithin*{equation}{subsection}



%% Свои команды
\DeclareMathOperator{\sgn}{\mathop{sgn}}

%% Перенос знаков в формулах (по Львовскому)
\newcommand*{\hm}[1]{#1\nobreak\discretionary{}
{\hbox{$\mathsurround=0pt #1$}}{}}

%%% Работа с картинками
\usepackage{graphicx}  % Для вставки рисунков
\graphicspath{{images/}{images2/}}  % папки с картинками
\setlength\fboxsep{3pt} % Отступ рамки \fbox{} от рисунка
\setlength\fboxrule{1pt} % Толщина линий рамки \fbox{}
\usepackage{wrapfig} % Обтекание рисунков и таблиц текстом

%%% Работа с таблицами
\usepackage{array,tabularx,tabulary,booktabs} % Дополнительная работа с таблицами
\usepackage{longtable}  % Длинные таблицы
\usepackage{multirow} % Слияние строк в таблице
\usepackage{graphicx}
\usepackage{fancyhdr}
\usepackage{hyperref}
\usepackage{booktabs}

\newcommand{\lt}{\left}
\newcommand{\rt}{\right}
\newcommand{\al}{\alpha}
\newcommand{\p}{\partial}
\newcommand{\D}{\Delta}
\newcommand{\fr}{\frac}
\newcommand{\dfr}{\dfrac}
\newcommand{\mbf}{\mathbf}


\pagestyle{fancy}
\fancyhf{}
\pagestyle{plain} % нумерация вкл.

\rhead{\today}
\lhead{Соколов Игорь, группа 573}

%%% Заголовок
\author{Соколов Игорь, группа 573}
\title{ДЗ 5 по Методам Оптимизации. \newline Векторное дифференцирование}
\date{\today}

\begin{document} % конец преамбулы, начало документа

\maketitle


\section{}

Найти $\nabla f(x)$, если $f(x) = \|Ax\| - \|x^TA\|$

\vspace{\baselineskip}

\textbf{Решение:}

\vspace{\baselineskip}

Перепишем в скалярном виде:

$$f(x) = \sqrt{\sum\limits_{i=1}^n\lt(\sum\limits_{j=1}^n a_{ij}x_j\rt)^2} - \sqrt{\sum\limits_{j=1}^n\lt(\sum\limits_{i=1}^n x_ia_{ij}\rt)^2}$$


\begin{multline}
\fr{\p f}{\p x_k} = \fr{\sum\limits_{i=1}^n 2a_{ik}\lt(\sum\limits_{j=1}^n a_{ij}x_j\rt)}{2\sqrt{\sum\limits_{i=1}^n\lt(\sum\limits_{j=1}^n a_{ij}x_j\rt)^2}}
- 
\fr{\sum\limits_{j=1}^n 2a_{kj}\lt(\sum\limits_{i=1}^n a_{ij}x_j\rt)}{2\sqrt{\sum\limits_{i=1}^n\lt(\sum\limits_{j=1}^n a_{ij}x_j\rt)^2}} =\\= 
\fr{\sum\limits_{i=1}^n a_{ik}\lt(\sum\limits_{j=1}^n a_{ij}x_j\rt)}{\sqrt{\sum\limits_{i=1}^n\lt(\sum\limits_{j=1}^n a_{ij}x_j\rt)^2}}
- 
\fr{\sum\limits_{j=1}^n a_{kj}\lt(\sum\limits_{i=1}^n a_{ij}x_j\rt)}{\sqrt{\sum\limits_{i=1}^n\lt(\sum\limits_{j=1}^n a_{ij}x_j\rt)^2}}
=\\= \fr{a_k^T\lt(Ax\rt)}{\|Ax\|} - \fr{a_k(x^TA)^T}{\|x^TA\|} = \fr{a_k^T\lt(Ax\rt)}{\|Ax\|} - \fr{a_k(A^Tx)}{\|x^TA\|} 
\end{multline}

$$\nabla f(x) = \fr{A^T\lt(Ax\rt)}{\|Ax\|} - \fr{A(A^Tx)}{\|x^TA\|} $$

\textbf{Ответ:} $\nabla f(x) = \dfr{A^T\lt(Ax\rt)}{\|Ax\|} - \dfr{A(A^Tx)}{\|x^TA\|} $

\section{}

Найти $\nabla f(x), f''(x)$, если $f(x) = \dfrac{-1}{1 + x^Tx}$

\vspace{\baselineskip}

\textbf{Решение:}

\vspace{\baselineskip}

Перепишем $f(x)$ в скаляром виде:


$$f(x) = \frac{-1}{1 + \sum\limits_{i=1}^n x_i^2} $$
$$\frac{\p f(x)}{\p x_k} = \frac{\p}{\p x_k}\lt(\frac{-1}{1 + \sum\limits_{i=1}^n x_i^2} \rt) = \frac{2x_k}{\lt(1 + \sum\limits_{i=1}^n x_i^2 \rt)^2}$$

$$\Rightarrow \nabla f(x) =  \frac{2{x}}{\lt(1 + {x}^T { x} \rt)^2}$$

Пусть $g_k = \dfrac{\p f(x)}{\p x_k} \rightarrow H_{k,p} = \dfrac{\p g_k}{\p x_p}$

Тогда 
\begin{multline}
H_{k,p} = \dfrac{\p^2 f(x)}{\p x_p \p x_k} = \frac{\p}{\p x_p}\lt(\frac{2x_k}{\lt(1 + \sum\limits_{i=1}^n x_i^2 \rt)^2}\rt) =\\= \frac{2\delta_{k,p}\lt(1 + \sum\limits_{i=1}^n x_i^2 \rt)^2 - 8x_kx_p\lt(1 + \sum\limits_{i=1}^n x_i^2 \rt)}{\lt(1 + \sum\limits_{i=1}^n x_i^2 \rt)^4} = \frac{2}{\lt(1 + \sum\limits_{i=1}^n x_i^2 \rt)^2}\lt(\delta_{k,p} - \frac{4x_p x_k}{1 + \sum\limits_{i=1}^n x_i^2} \rt)
\end{multline}

$$\Rightarrow f''(x) =  \frac{2}{\lt(1 + x^Tx\rt)^2}\lt(\mathbb{E} - \frac{4xx^T}{1 + x^Tx} \rt) $$

\textbf{Ответ:}

$\nabla f(x) =  \dfrac{2{x}}{\lt(1 + {x}^T { x} \rt)^2}$

$f''(x) =  \dfrac{2}{\lt(1 + x^Tx\rt)^2}\lt(\mathbb{E} - \dfrac{4xx^T}{1 + x^Tx} \rt)$

%3

\section{}\label{ref_task_3}

Найти $f'(X)$, если $f(X) = \det X$

{\bf Примечание:} здесь под $f'(X)$ подразумевается оценка фунции $f(X)$ первого порядка в смысле разложения в ряд Тейлора:
     $$f(X + \Delta X) \approx f(X) + \mathbf{tr}(f'(X)^T \Delta X)$$


\vspace{\baselineskip}

\textbf{Решение:}

\vspace{\baselineskip}

Пусть $g(X) = \log\det X$

Найдем $g(X + \Delta X) = \log\det X$ , где $\Delta X$ - некоторая малая матрица по норме Фробениуса ( $ $).

Тогда $f(X + \Delta X) = e^{g(X + \Delta X)} = \det (X + \D X)$

\vspace{\baselineskip}

Вспомним, что производная имеет смысл линейной аппроксимация функции в окрестности точки.

Заметим, что:
 \begin{multline}
 \log\det\left[ X+ \Delta X\right] = \log \det \left[ X^{1/2} \left(I + X^{-1/2} \Delta X X^{-1/2}\right)X^{1/2}\right] =\\= \log \det \left[ X^{1/2} \right]\det \left[ I + X^{-1/2} \Delta X X^{-1/2}\right] \det \left[ X^{1/2}\right] =\\= \log \det \left[ X \right]\det \left[ I + X^{-1/2} \Delta X X^{-1/2}\right]=\\= \log \det \left[ X \right] + \log\det \left[ I + X^{-1/2} \Delta X X^{-1/2}\right]
 \end{multline}
 
Известно, что {\it определитель матрицы равен произведению её собственных значений}
 
 $$\log\det\left[ X+ \Delta X\right] = \log\det X + \sum\limits_{i=1}^n \log(1 + \lambda_i)$$
 
Где $\lambda_i$ - собственные числа матрицы $X^{-1/2} \Delta X X^{-1/2}$. Далее используем факт "малости" матрицы $\Delta X$ (в смысле нормы этой матрицы), следовательно для приближения первого порядка справедливо: $\log (1 + \lambda_i) \approx \lambda_i$ т.к. $\lambda_i$ так же должны быть малыми.
 

 $$\log\det\left[ X+ \Delta X\right] \approx \log\det X + \sum\limits_{i=1}^n \lambda_i$$
 
 Также пользуемся свойством следа матрицы: {\it след матрицы равен сумме её собственных значений}.
 
 \begin{multline}\label{eq_log_det}
 \log\det\left[ X+ \Delta X\right] \approx \log\det X + \mathbf{tr}\left[X^{-1/2} \Delta X X^{-1/2}\right] =\\= \log\det X + \mathbf{tr}\left[X^{-1/2} X^{-1/2} \Delta X \right] = \log\det X + \mathbf{tr}\left[X^{-1}\Delta X \right]
 \end{multline}
 
 Так как по условию матрица $X$ - квадратная, симметричная, то $\lt(X^{-1}\rt)^T = \lt(X^{T}\rt)^{-1} = X^{-1}$.
 
 Заметим, что след произведения матриц есть обобщения скалярного произведения на пространство матриц. 
 
След матрицы также обладает следующий свойством:
 
 $$\mathbf{tr}(A^TB) = \mathbf{tr}(AB^T) = \mathbf{tr}(B^TA) = \mathbf{tr}(BA^T)$$. 
 
В силу всех вышеприведенных фактов, справедливо:
 
 \begin{equation}\label{eq_g}
 g(X + \Delta X) \approx g(X) + \langle g'(X), \Delta X \rangle
 \end{equation}
 
\begin{equation}\label{eq_dg}
\Rightarrow g'(X) = X^{-1}
\end{equation}
 
 Из \eqref{eq_log_det} $\Rightarrow$ $$e^{\log\det\left[ X+ \Delta X\right]} = e^{\log\det X + \mathbf{tr}\left[X^{-1}\Delta X \right]} $$
 $$\det\left[ X+ \Delta X\right] = e^{\log\det X} e^{\mathbf{tr}\left[X^{-1}\Delta X \right]} $$
 
 В силу малости матрицы $\D X$ справедливо $e^{\mathbf{tr}\left[X^{-1}\Delta X \right]} = 1 + \mathbf{tr}\left[X^{-1}\Delta X \right] $
 
 $$\Rightarrow\det\left[ X+ \Delta X\right] = \det X\lt( 1 + \mathbf{tr}\left[X^{-1}\Delta X \right]\rt)  $$
 \begin{equation}\label{eq_3}
 \det\left[ X+ \Delta X\right] = \det X+ \det X\mathbf{tr}\left[X^{-1}\Delta X \right]
 \end{equation}
 
 Заметим, что $det X$ - скаляр и в силу свойства линейности следа матрицы $$\mathbf{tr}\lt(\al A \rt) = \al \mathbf{tr}\lt(A \rt)$$
 
 Получаем 
 
 $$ \det\left[ X+ \Delta X\right] = \det X+ \mathbf{tr}\left[(\det X) X^{-1}\Delta X \right]$$
 
 Пользуясь аналогичными примеру \eqref{eq_g} рассуждениями, получаем:
 
 $$ \det\left[ X+ \Delta X\right] = \det X+ \mathbf{tr}\langle (\det X) X^{-1}, \Delta X \rangle$$
 
Значит, $f'(X) = det(X) X^{-1}$

\vspace{\baselineskip}

\textbf{Ответ:} $f'(X) = det(X) X^{-1}$

%4
\section{}

Найти $f''(X)$, если $f(X) = \log \det X$

{\bf Примечание:} здесь под $f''(X)$ подразумевается оценка фунции $f(X)$ второго порядка в смысле разложения в ряд Тейлора:
     $$f(X + \Delta X) \approx f(X) + \mathbf{tr}(f'(X)^T \Delta X) + \frac{1}{2}\mathbf{tr}(\Delta X^T f''(X) \Delta X)$$

\vspace{\baselineskip}

\textbf{Решение:}

\vspace{\baselineskip}

Задача решается аналогично задаче \ref{ref_task_3}.

Воспользуемся примечанием к задаче.

\begin{multline}
\log\det\left[ X+ \Delta X\right] = \log \det \left[ X^{1/2} \left(I + X^{-1/2} \Delta X X^{-1/2}\right)X^{1/2}\right] =\\= \log \det \left[ X^{1/2} \right]\det \left[ I + X^{-1/2} \Delta X X^{-1/2}\right] \det \left[ X^{1/2}\right] =\\= \log \det \left[ X \right]\det \left[ I + X^{-1/2} \Delta X X^{-1/2}\right]=\\= \log \det \left[ X \right] + \log\det \left[ I + X^{-1/2} \Delta X X^{-1/2}\right]
\end{multline}

Известно, что {\it определитель матрицы равен произведению её собственных значений}

$$\log\det\left[ X+ \Delta X\right] = \log\det X + \sum\limits_{i=1}^n \log(1 + \lambda_i)$$

Где $\lambda_i$ - собственные числа матрицы $X^{-1/2} \Delta X X^{-1/2}$. Далее используем факт "малости" матрицы $\Delta X$ (в смысле нормы этой матрицы), следовательно для приближения второго порядка справедливо: $\log (1 + \lambda_i) \approx \lambda_i - \dfr{\lambda_i^2}{2}$ т.к. $\lambda_i$ так же должны быть малыми.

$$\log\det\left[ X+ \Delta X\right] \approx \log\det X + \sum\limits_{i=1}^n \lambda_i - \fr{1}{2}\sum\limits_{i=1}^n \lambda_i^2$$

{\bf Докажем}, что $\mathbf{tr}(XX) = \sum\limits_{i=1}^n \lambda_i^2$, где $\lambda_i$ - собственные значения матрицы $X$, а $XX = X^2$.

\begin{proof}

Пусть $S$ - матрица перехода из исходного базиса матрицы $X$ к базису, в котором матрица $X$ диагональна и на её диагонали стоят собственные значения (столбцы $S$ есть собственные вектора матрицы $X$ записанные в старом базисе)

Тогда $X_{eigen} = S^{-1}XS$

Воспользуемся свойством циклической перестановки: 

$\mathbf{tr}(ABC) = \mathbf{tr}(BCA) =\mathbf{tr}(CAB)$

$\mathbf{tr}(X_{eigen}) = \sum\limits_{i=1}^n \lambda_i = \mathbf{tr}(S^{-1}XS)= \mathbf{tr}(SS^{-1}X)= \mbf{tr}(X)$ (доказали ранее сформулированное свойство)

Воспользуемся еще одним свойством следа матрицы: 

$ \mbf{tr}(X^TY) = \sum\limits_{i,j} X_{i,j}Y_{i,j}$

В нашем случае:

$ \mbf{tr}(X^TX) = \mbf{tr}(XX) = \sum\limits_{i,j} X_{i,j}X_{i,j}$

$\Rightarrow \mbf{tr}(X_{eigen}X_{eigen}) = \sum\limits_{i=1}^n  \lambda_i^2$

Но $\mbf{tr}(X_{eigen}X_{eigen}) = \mbf{tr}(S^{-1}XSS^{-1}XS) = \mbf{tr}(S^{-1}XXS) = \mbf{tr}(SS^{-1}XX) = \mbf{tr}(XX)$ 

$\Rightarrow \mbf{tr}(XX) =\sum\limits_{i=1}^n \lambda_i^2$
\end{proof}
\begin{multline}\label{eq_log_det}
\log\det\left[ X+ \Delta X\right] \approx \log\det X + \mathbf{tr}\left[X^{-1/2} \Delta X X^{-1/2}\right] - \fr{1}{2}\mbf{tr}\lt[X^{-1/2} \Delta X X^{-1/2}X^{-1/2} \Delta X X^{-1/2}\rt]
=\\= 
\log\det X + \mathbf{tr}\left[X^{-1/2} X^{-1/2} \Delta X \right] - \fr{1}{2}\mbf{tr}\lt[X^{-1} \Delta X X^{-1} \Delta X \rt] =\\= \log\det X + \mathbf{tr}\left[\lt(X^{-1}\rt)^T\Delta X \right] + \fr{1}{2}\mbf{tr}\lt[\Delta X^T \lt(-X^{-1} X^{-1}\rt) \Delta X \rt]
\end{multline}

В примере \eqref{eq_log_det} воспользовались тем что $X$ - квадратная и симметричная.

$\Rightarrow f''(X) = -\lt(X^{-1}\rt)^2$

\vspace{\baselineskip}

\textbf{Ответ:} $f''(X) = -\lt(X^{-1}\rt)^2$

%5
\section{}

Найти градиент и гессиан функции $f : \mathbb{R}^n \to \mathbb{R}$, $f(x) = \log \sum\limits_{i=1}^m \exp (a_i^Tx + b_i), \;\;\;\; a_1, \ldots, a_m \in \mathbb{R}^n; \;\;\;  b_1, \ldots, b_m  \in \mathbb{R}$

\vspace{\baselineskip}

\textbf{Решение:}

\begin{equation}
\fr{\p f}{x_k} = \fr{\sum\limits_{i=1}^m a_{ik} \exp \lt(\sum\limits_{j=1}^n a_{ij}x_j + b_i\rt)}{\sum\limits_{i=1}^m \exp \lt(\sum\limits_{j=1}^n a_{ij}x_j + b_i\rt)}
\end{equation}

\begin{equation}
\nabla f = \fr{\sum\limits_{i=1}^m a_{i} \exp \lt(a_{i}^Tx + b_i\rt)}{\sum\limits_{i=1}^m \exp \lt(a_{i}^Tx + b_i\rt)}
\end{equation}

Пусть $g_k = \dfr{\p f}{\p x_k}$

$H_{k,p} = \dfr{\p g}{\p x_p}$

Тогда 

\begin{multline}
H_{k,p} = \fr{\p g}{\p x_p} = \fr{\p^2 f}{\p x_p \p x_k} =\\= \fr{\sum\limits_{i=1}^m a_{ik}a_{ip} \exp \lt(\sum\limits_{j=1}^n a_{ij}x_j + b_i\rt)}{\sum\limits_{i=1}^m \exp \lt(\sum\limits_{j=1}^n a_{ij}x_j + b_i\rt)} - \fr{\lt[\sum\limits_{i=1}^m a_{ik} \exp \lt(\sum\limits_{j=1}^n a_{ij}x_j + b_i\rt)\rt]\lt[\sum\limits_{i=1}^m a_{ik} \exp \lt(\sum\limits_{j=1}^n a_{ip}x_j + b_i\rt) \rt]}{\lt[\sum\limits_{i=1}^m \exp \lt(\sum\limits_{j=1}^n a_{ij}x_j + b_i\rt)\rt]^2}
\end{multline}

$$f''(x) = \fr{\sum\limits_{i=1}^m a_{i}a_{i}^T \exp \lt( a_{i}^Tx + b_i\rt)}{\sum\limits_{i=1}^m \exp \lt(a_{i}^Tx + b_i\rt)} - \fr{\lt[\sum\limits_{i=1}^m a_{i} \exp \lt( a_{i}^Tx + b_i\rt)\rt]\lt[\sum\limits_{i=1}^m a_{i}^T \exp \lt(a_{i}^Tx + b_i\rt) \rt]}{\lt[\sum\limits_{i=1}^m \exp \lt(a_{i}^Tx + b_i\rt)\rt]^2}$$

\textbf{Ответ:} 

$\nabla f = \dfr{\sum\limits_{i=1}^m a_{i} \exp \lt(a_{i}^Tx + b_i\rt)}{\sum\limits_{i=1}^m \exp \lt(a_{i}^Tx + b_i\rt)}$

$f''(x) = \dfr{\sum\limits_{i=1}^m a_{i}a_{i}^T \exp \lt( a_{i}^Tx + b_i\rt)}{\sum\limits_{i=1}^m \exp \lt(a_{i}^Tx + b_i\rt)} - \dfr{\lt[\sum\limits_{i=1}^m a_{i} \exp \lt( a_{i}^Tx + b_i\rt)\rt]\lt[\sum\limits_{i=1}^m a_{i}^T \exp \lt(a_{i}^Tx + b_i\rt) \rt]}{\lt[\sum\limits_{i=1}^m \exp \lt(a_{i}^Tx + b_i\rt)\rt]^2}$


\end{document} % конец документа

