% !TeX spellcheck = <none>
\documentclass[a4paper,12pt]{article}

%%% Работа с русским языком
\usepackage{cmap}					% поиск в PDF
\usepackage{mathtext} 				% русские буквы в формулах
\usepackage[T2A]{fontenc}			% кодировка
\usepackage[utf8]{inputenc}			% кодировка исходного текста
\usepackage[english,russian]{babel}	% локализация и переносы
\usepackage{comment}


%%% Дополнительная работа с математикой
\usepackage{amsfonts,amssymb,amsthm,mathtools} % AMS
\usepackage{amsmath}
\usepackage{icomma} % "Умная" запятая: $0,2$ --- число, $0, 2$ --- перечисление

%% Номера формул
%\mathtoolsset{showonlyrefs=true} % Показывать номера только у тех формул, на которые есть \eqref{} в тексте.

%% Шрифты
\usepackage{euscript}	 % Шрифт Евклид
\usepackage{mathrsfs} % Красивый матшрифт

\usepackage{extsizes} % Возможность сделать 14-й шрифт
\usepackage{geometry} % Простой способ задавать поля
\geometry{top=25mm}
\geometry{bottom=35mm}
\geometry{left=20mm}
\geometry{right=20mm}

\usepackage{chngcntr}
\usepackage{hyperref}

\usepackage{setspace} % Интерлиньяж
%\onehalfspacing % Интерлиньяж 1.5
%\doublespacing % Интерлиньяж 2
%\singlespacing % Интерлиньяж 1

\usepackage{lastpage} % Узнать, сколько всего страниц в документе.
\usepackage{soulutf8} % Модификаторы начертания

\counterwithin*{equation}{section}
\counterwithin*{equation}{subsection}


%% Свои команды
\DeclareMathOperator{\sgn}{\mathop{sgn}}

%% Перенос знаков в формулах (по Львовскому)
\newcommand*{\hm}[1]{#1\nobreak\discretionary{}
{\hbox{$\mathsurround=0pt #1$}}{}}

%%% Работа с картинками
\usepackage{graphicx}  % Для вставки рисунков
\graphicspath{{images/}{images2/}}  % папки с картинками
\setlength\fboxsep{3pt} % Отступ рамки \fbox{} от рисунка
\setlength\fboxrule{1pt} % Толщина линий рамки \fbox{}
\usepackage{wrapfig} % Обтекание рисунков и таблиц текстом

%%% Работа с таблицами
\usepackage{array,tabularx,tabulary,booktabs} % Дополнительная работа с таблицами
\usepackage{longtable}  % Длинные таблицы
\usepackage{multirow} % Слияние строк в таблице
\usepackage{graphicx}
\usepackage{fancyhdr}
\usepackage{hyperref}
\usepackage{booktabs}

\newenvironment{solution}
{\renewcommand\qedsymbol{$\blacksquare$}\begin{proof}[Solution]}{\end{proof}}

\newcommand{\lt}{\left}
\newcommand{\rt}{\right}
\newcommand{\al}{\alpha}
\newcommand{\p}{\partial}
\newcommand{\D}{\Delta}
\newcommand{\fr}{\frac}
\newcommand{\dfr}{\dfrac}
\newcommand{\mbf}{\mathbf}
\newcommand{\bb}{\mathbb}
\newcommand{\wt}{\widetilde}
\newcommand{\La}{\Lambda}
\newcommand{\la}{\lambda}
\newcommand{\opn}{\operatorname}
\newcommand{\vp}{\varphi}
\newcommand{\rw}{\rightarrow}
\newcommand{\iy}{\infty}
\newcommand{\Rw}{\Rightarrow}

\pagestyle{fancy}
\fancyhf{}
\pagestyle{plain} % нумерация вкл.

\rhead{\today}
\lhead{Соколов Игорь, группа 573}

%%% Заголовок
\author{Соколов Игорь, группа 573}
\title{ДЗ 1 по Методам Оптимизации.}
\date{\today}

\begin{document} % конец преамбулы, начало документа

\maketitle

\begin{enumerate}
\item $r_k = \lt\{\lt(0.707\rt)^k\rt\}_{k=1}^{\infty}$
\begin{solution}
	Тест корней:
	$\al =  \lim\limits_{k \rw \infty}\sup r_k^{1/k}$
	
	$\al = \lim\limits_{k \rw \infty}\sup \lt(0.707^k\rt)^{1/k} = 0.707$
	
	$\Rw \left\{ r_k\right\}_{k=1}^{\iy}$ имеет линейную сходимость с константной $\al = 0.707$
	
\end{solution}

\item $r_k =
 \lt\{\lt(0.707\rt)^{2^k}\rt\}_{k=1}^{\infty}$
\begin{solution}
	Тест отношений:
	$\al = \lim\limits_{k \rw \infty} \dfrac{r_{k+1}}{r_k}$
	
	\begin{multline}
		\lim\limits_{k \rw \infty} \dfrac{\lt(0.707\rt)^{2^{k+1}}}{\lt(0.707\rt)^{2^k}} =\lim\limits_{k \rw \infty} \dfr{\exp\ln\lt(0.707\rt)^{2^{k+1}}}{\exp\ln\lt(0.707\rt)^{2^{k}}} =\lim\limits_{k \rw \infty} \dfr{\exp(2^{k+1}\ln(0.707))}{\exp(2^{k}\ln(0.707))} =\\= \lim\limits_{k \rw \infty} \exp (2^k \ln(0.707)) = 0
	\end{multline}
	
	$\ln 0.707 < 0$ 
	
	$\Rw \left\{ r_k\right\}_{k=1}^{\iy}$ имеет сверхлинейную сходимость.
	
\end{solution}

\item
 $r_k = \lt\{\dfr{1}{k^2}\rt\}_{k=1}^{\infty}$
 \begin{solution}
 Тест отношений:
 $\lim\limits_{k \rw \infty} \dfrac{k^2}{(k+1)^2} = 1$
 
 $\Rw \left\{ r_k\right\}_{k=1}^{\iy}$ имеет сублинейную сходимость.
 \end{solution}

\item $r_k = \lt\{\dfr{1}{k!}\rt\}_{k=1}^{\infty}$
\begin{solution}
Тест отношений:
$\lim\limits_{k \rw \infty} \dfrac{k!}{(k+1)!} = \lim\limits_{k \rw \infty}\dfr{1}{k+1} = 0$

 $\Rw \left\{ r_k\right\}_{k=1}^{\iy}$ имеет сверхлинейную сходимость.
\end{solution}
\item 
$
\begin{cases}
\dfrac{1}{k},& \text{if $k$ is even}\\
\dfrac{1}{k^2},& \text{if $k$ is odd}
\end{cases}, 
$
\begin{solution}
	Тест отношений:
	$\lim\limits_{k \rw \infty}\inf \dfr{r_{k+1}}{r_k} = \lim\limits_{k \rw \infty} \dfr{k^2}{(k+1)^2} = 1$
	
	$\Rw \left\{ r_k\right\}_{k=1}^{\iy}$ имеет сублинейную сходимость.
	
\end{solution}
\item 
$
\begin{cases}
\dfrac{1}{k^k},& \text{if $k$ is even}\\
\dfrac{1}{k^{2k}},& \text{if $k$ is odd}
\end{cases}, 
$
\begin{solution}
Тест корней: 	
$\lim\limits_{k \rw \infty}\sup r_k^{1/k} = \lim\limits_{k \rw \infty}\dfr{1}{k} = 0	$

 $\Rw \left\{ r_k\right\}_{k=1}^{\iy}$ имеет сверхлинейную сходимость.
\end{solution}

\end{enumerate}


\end{document} % конец документа

