% Этот шаблон документа разработан в 2014 году
% Данилом Фёдоровых (danil@fedorovykh.ru) 
% для использования в курсе 
% <<Документы и презентации в LaTeX>>, записанном НИУ ВШЭ
% для Coursera.org: http://coursera.org/course/latex .
% Исходная версия шаблона --- 
% https://www.writelatex.com/coursera/latex/1.2

\documentclass[a4paper,12pt]{article} % добавить leqno в [] для нумерации слева

%%% Работа с русским языком
\usepackage{cmap}					% поиск в PDF
\usepackage{mathtext} 				% русские буквы в формулах
\usepackage[T2A]{fontenc}			% кодировка
\usepackage[utf8]{inputenc}			% кодировка исходного текста
\usepackage[english,russian]{babel}	% локализация и переносы

%%% Дополнительная работа с математикой
\usepackage{amsmath,amsfonts,amssymb,amsthm,mathtools} % AMS
\usepackage{icomma} % "Умная" запятая: $0,2$ --- число, $0, 2$ --- перечисление

%% Номера формул
%\mathtoolsset{showonlyrefs=true} % Показывать номера только у тех формул, на которые есть \eqref{} в тексте.

%% Шрифты
\usepackage{euscript}	 % Шрифт Евклид
\usepackage{mathrsfs} % Красивый матшрифт

%% Свои команды
\DeclareMathOperator{\sgn}{\mathop{sgn}}

%% Перенос знаков в формулах (по Львовскому)
\newcommand*{\hm}[1]{#1\nobreak\discretionary{}
{\hbox{$\mathsurround=0pt #1$}}{}}

%%% Заголовок
\author{Бекезин Никита}
\title{4 д/з по курсу методы оптимизации}
\date{\today}

%%% Работа с картинками
\usepackage{graphicx}  % Для вставки рисунков
\graphicspath{{images/}{images2/}}  % папки с картинками
\setlength\fboxsep{3pt} % Отступ рамки \fbox{} от рисунка
\setlength\fboxrule{1pt} % Толщина линий рамки \fbox{}
\usepackage{wrapfig} % Обтекание рисунков и таблиц текстом

\usepackage{geometry} % Простой способ задавать поля
\geometry{top=25mm}
\geometry{bottom=35mm}
\geometry{left=20mm}
\geometry{right=20mm}

\begin{document} % конец преамбулы, начало документа

\maketitle


\begin{center}
	\textbf{Задача №1}
\end{center}

\begin{enumerate}
	\item[] Найти и изобразить на плоскости множество, сопряженное к многогранному конусу:
	$ \textbf{S} = \mathbf{conv}\{(-4, -1), (-2, -1), (-2, 1)\} + \mathbf{cone}\{(1, 0), (2, 1) \}$
	\item[] \textbf{Решение}
		\begin{enumerate}
			\item[] Воспользуемся \textbf{теоремой}:
			
			Пусть $x_1,\ldots,x_m \in\mathbb{R}^n$. Сопряженным к многогранному множеству:\\ $$ \textbf{S} = \mathbf{conv}\{x_1, \ldots, x_k\} + \mathbf{cone}\{x_{k+1}, \ldots, x_m \}$$
			является полиэдр (многогранник):
			\[\textbf{S} = \left\{p \in \mathbb{R}^n \mid \langle{ p, x_i }\rangle \ge -1, i = \overline{1, k}; \langle{ p, x_i }\rangle \ge 0 , i =\overline{k+1, m} \right\} \]
			
			Используя теорему:
			\[\textbf{S*} = \{p \in \mathbb{R}^n \mid -4p_1 - p_2 \ge -1\textbf{,} - 2p_1 -p_2 \ge -1\textbf{,} -2p_1 + p_2 \ge -1 \textbf{,} p_1 \ge 0 \textbf{,} 2p_1 + p_2 \ge 0\}\]
			
			Запишем условия в виде системы:
			\[
			\left\{
			\begin{aligned}
			p_2 &\leq -4p_1 + 1  \\
			p_2 &\leq -2p_1 + 1\\
			p_2 &\ge 2p_1 - 1\\
			p_1 &\ge 0  \\
			p_2 &\ge -2p_1
			\end{aligned}
			\right.
			\]
			
		%	\includegraphics[scale=0.05]{graph1.jpg}
		\includegraphics[scale=0.2]{graph1.jpg}
			
		\end{enumerate}
\end{enumerate}

\end{document}